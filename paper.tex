\documentclass{article}

\usepackage[utf8]{inputenc}
\usepackage{amsmath}
\usepackage{amsthm}

\theoremstyle{definition}
\newtheorem{definition}{Definition}

\title{What is the probability of you answering this question correctly?}
\author{Vishnu Vivek Nittoor}
\date{July 19 2021}

\begin{document}
\maketitle

\section{Introduction: Unpacking the `Question'}
\label{introduction}

Within this paper, the primary question named in the title (``What is the probability of you answering this question correctly?") is referred to as $Q_0$. As we encounter other related questions, we will define them using similar notation. The primary goal of this paper is to develop a deeper understanding of $Q_0$.

Let us define $P: \left[0, 1\right] \rightarrow \left[0, 1\right] $ as follows:

\begin{definition}
    $P(x)$ represents the probability that $x$ is the correct answer to $Q_0$.
\end{definition}

Using this definition, we can proceed to construct a definition of an answer to $Q_0$:
\begin{definition}
    $\forall a \in \left[0, 1\right],  [P(a) = a] \implies$ a is a correct answer.
\end{definition}

\section{Consistent Answers}

A set of sentences is logically consistent if and only if it is possible for all the members of that set to be true. Since the only statement that an answer $a$ needs to satisfy is $P(a) = a$, we can construct a definition for what it means for an answer to be \textit{consistent}.

\begin{definition}
    An answer $a$ is consistent if it is \textbf{not impossible} for $P(a) = a$.
\end{definition}

\section{A Brief Exploration of Related Questions}

\section{The Empirical Approach}

\section{The Oscillating Answer - A Computational Approach To Truth}

\section{Good Answers}

\section{Is there an answer?}

\section{What is the probability of YOU answering this question correctly?}

\end{document}